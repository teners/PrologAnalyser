%\NeedsTeXFormat{LaTeX2e}
\documentclass[a4paper,12pt]{article} 

\usepackage{lipsum}
\usepackage{fancyhdr}
%% set indent at the beginning of a paragraph
\usepackage{indentfirst} 
\usepackage[T2A]{fontenc}
\usepackage[utf8]{inputenc}	
\usepackage[russian]{babel}
\usepackage[a4paper]{geometry} 
\usepackage[titletoc]{appendix}
%% set urls' colors 
\usepackage[colorlinks=true,urlcolor=black,linkcolor=black,filecolor=black,citecolor=black]{hyperref}

\geometry{top=3cm}
\geometry{left=4cm}
\geometry{bottom=3cm}
\geometry{right=1.5cm}

%% macro starts new section on new page
\newcommand{\psection}[1]{\newpage\section{#1}}
%% this one does the same but for unenumerate sections
\newcommand{\unsection}[2]{\newpage\section*{#2}\addcontentsline{toc}{section}{#2}}

\begin{document}
    \begin{titlepage}
        \thispagestyle{fancy}
        \fancyhf{}  
        \renewcommand{\headrulewidth}{0pt} 
        \newgeometry{top=20mm,bottom=20mm,left=20mm,right=15mm,includeheadfoot,headheight=105pt}
        
        %% header 
        \chead{{\large 
        Министерство образования и науки Российской Федерации\\ 
        Федеральное государственное автономное образовательное учреждения высшего образования
        <<Санкт-Петербургский государственный университет аэрокосмического приборостроения>>}
        \vskip 1em
        {\large Кафедра №43\\}}
        
        %% footer
        \cfoot{\large Санкт-Петербург\\ \the\year} 


        \begin{flushleft}
            КУРСОВАЯ РАБОТА\\
            ЗАЩИЩЕНА С ОЦЕНКОЙ\\
            \vskip 1em
            РУКОВОДИТЕЛЬ\\
        \end{flushleft}

        \vskip 1em 
        \setlength{\tabcolsep}{0.8cm}
        {\raggedright\begin{tabular}{c c c c c}
            доцент, к.т.н.                          & &                      & & Максимова Т.~М. \\ \cline{1-1} \cline{3-3} \cline{5-5}  
            \tiny{должность, уч. степень, звание} & & \tiny{подпись, дата} & & \tiny{инициалы, фамилия} 
        \end{tabular}} 

        \begin{center} \large 
            \vskip 0.15\textheight
            Пояснительная записка к курсовой работе\\
            \vskip 1em
            {\sc Синтаксически управляемый анализ анализ\\ семантики идентификаторов в текстах программ}\\
            по дисциплине: теория языков программирования\\ и методы трансляции
        \end{center}

        \begin{flushleft}
           \vskip 0.2\textheight
           РАБОТУ ВЫПОЛНИЛ\\ 
        \end{flushleft}

        \setlength{\tabcolsep}{0.8cm}
        {\raggedright\begin{tabular*}{450pt}{@{\extracolsep{\fill}} l c c c c c}
                СТУДЕНТ ГР. & 4331 & &                      & & Соколов С.~А. \\ \cline{2-2} \cline{4-4} \cline{6-6}
                            &      & & \tiny{подпись, дата} & & \tiny{фамилия, инициалы}
        \end{tabular*}}

    \end{titlepage}

    \tableofcontents 

%% content

    \unsection*{Введение}
    %% lorem ipsum, replace with your own text
    \lipsum[3-6]    

    \psection{Постановка задачи}
    \lipsum[7-8]
    
    \psection{Описание исходного языка}
    \lipsum[10-14]
\end{document}
