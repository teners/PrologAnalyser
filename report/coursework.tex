%\NeedsTeXFormat{LaTeX2e}
\documentclass[a4paper,12pt]{article} 

\usepackage{lipsum}
\usepackage{fancyhdr}
%% set indent at the beginning of a paragraph
\usepackage{indentfirst} 
\usepackage[T2A]{fontenc}
\usepackage[utf8]{inputenc}	
\usepackage[russian]{babel}
\usepackage[a4paper]{geometry} 
\usepackage[titletoc]{appendix}
%% set urls' colors 
\usepackage{enumitem}
\usepackage{listings}
\usepackage[colorlinks=true,urlcolor=black,linkcolor=black,filecolor=black,citecolor=black]{hyperref}

%% title page settings
%% just put your values in these 'variables'
%% --------------------------------------------------------    
\newcommand{\chair}         {№43}
\newcommand{\degree}        {доцент, к.т.н.}
\newcommand{\professorname} {Максимова Т.~М.}
\newcommand{\reporttype}    {Пояснительная записка к}
\newcommand{\worktype}      {курсовой работе}
\newcommand{\worktitle}     {Синтаксически управляемый анализ анализ\\ семантики идентификаторов в текстах программ}
\newcommand{\subject}       {теория языков программирования\\ и методы трансляции}  
\newcommand{\group}         {4331}
\newcommand{\studentname}   {Соколов С.~А.}
%% --------------------------------------------------------    


\geometry{top=3cm}
\geometry{left=4cm}
\geometry{bottom=3cm}
\geometry{right=1.5cm}

%% macro starts new section on new page
\newcommand{\psection}[1]{\newpage\section{#1}}
%% this one does the same but for unenumerate sections
\newcommand{\unsection}[1]{\newpage\section*{#1}\addcontentsline{toc}{section}{#1}}

\begin{document}
    \begin{titlepage}
        \thispagestyle{fancy}
        \fancyhf{}  
        \renewcommand{\headrulewidth}{0pt} 
        \newgeometry{top=20mm,bottom=20mm,left=20mm,right=15mm,includeheadfoot,headheight=105pt}
        
        %% header 
        \chead{{\large 
        Министерство образования и науки Российской Федерации\\ 
        Федеральное государственное автономное образовательное учреждения высшего образования
        <<Санкт-Петербургский государственный университет аэрокосмического приборостроения>>}
        \vskip 1em
        {\large Кафедра \chair\\}}
        
        %% footer
        \cfoot{\large Санкт-Петербург\\ \the\year} 


        \begin{flushleft}
            РАБОТА\\
            ЗАЩИЩЕНА С ОЦЕНКОЙ\\
            \vskip 1em
            РУКОВОДИТЕЛЬ\\
        \end{flushleft}

        \vskip 1em 
        \setlength{\tabcolsep}{0.8cm}
        {\raggedright\begin{tabular}{c c c c c}
            \degree                               & &                      & & \professorname \\ \cline{1-1} \cline{3-3} \cline{5-5}  
            \tiny{должность, уч. степень, звание} & & \tiny{подпись, дата} & & \tiny{инициалы, фамилия} 
        \end{tabular}} 

        \begin{center} \large 
            \vskip 0.15\textheight
            \reporttype~\worktype\\
            \vskip 1em
            {\sc \worktitle}\\
            по дисциплине:~\subject
        \end{center}

        \begin{flushleft}
           \vskip 0.2\textheight
           РАБОТУ ВЫПОЛНИЛ\\ 
        \end{flushleft}

        \setlength{\tabcolsep}{0.8cm}
        {\raggedright\begin{tabular*}{450pt}{@{\extracolsep{\fill}} l c c c c c}
                СТУДЕНТ ГР. & \group & &                      & & \studentname \\ \cline{2-2} \cline{4-4} \cline{6-6}
                            &        & & \tiny{подпись, дата} & & \tiny{фамилия, инициалы}
        \end{tabular*}}

    \end{titlepage}

    \tableofcontents 

%% content
%% --------------------------------------------------------
    \unsection{Введение}
    %\addcontentsline{toc}{section}{Введение}
   
    Синтаксический и семантический анализ являются очень важными этапами процесса компиляции.
	Синтаксический анализ рассматривает цепочку лексем и проверяет, удовлетворяет ли она
	структурным условиям сформулированным в описании грамматики языка.
	Семантический анализ определяет смысл базовых конструцкий языка. Данный процесс синтаксически управляем,
	т.е. когда синтаксический анализатор распознает определенную конструкцию языка, для нее вызывается соответствующее
	семантическое правило. Также стоит отметить, что во время этого процесса строится абстрактное синтаксическое дерево разбора,
	по которому в последствии получается промежуточный код, который будет требоваться на последующий стадиях компиляции.\\
	\indent Согласно варианту, анализируемый язык программирования -- подмножество GNU Prolog.\\
    \indent Prolog — язык и система логического программирования, основанные на языке предикатов математической 
    логики дизъюнктов Хорна, представляющей собой подмножество логики предикатов первого порядка.
    Язык сосредоточен вокруг небольшого набора основных механизмов, включая сопоставление с образцом, древовидного 
    представления структур данных и автоматического перебора с возвратами. Хорошо подходит для решения задач, где 
    рассматриваются объекты (в частности структурированные объекты) и отношения между ними. Пролог, благодаря своим 
    особенностям, используется в области искусственного интеллекта, компьютерной лингвистики и нечислового 
    программирования в целом.

    \psection{Постановка задачи}
    \label{sec:task} 
    Задачей является разработка синтаксического и семантического анализатора подмножества Prolog. Синтаксический
    анализ должен осуществляться с помощью конечного детерминированного автомата с магазинной памятью. Семантический
    анализ должен быть основан на грамматике свойств.\\
    Рассматриваемое подмножество языка:
    \begin{itemize}[leftmargin=*] \itemsep1pt
        \item[---] программа должна состоять из списка правил и списка целей;
        \item[---] цель всегда начинается с ключевого слова <<:-init>>, в круглых скобках указывается цель и её аргументы;
        \item[---] правила должны начинаться со строчной буквы, состоять из списка условий, разделённых символом <<запятая>> 
            и оканчиваться символом <<точка>>;
        \item[---] могут быть использованы встроенные предикаты <<write>>, <<ln>>, <<succ>>, а так же, ключевое слово <<is>>;
        \item[---] переменные должны начинаться с заглавной буквы;
        \item[---] в список аргументов правила или цели могут входить не только переменные, но и целые числа, строки, списки.
    \end{itemize}
    Анализируемые семантические ошибки:
    \begin{itemize}[leftmargin=*] \itemsep1pt
        \item[---] использование необъявленного правила;
        \item[---] неверное количество аргументов в правиле.
    \end{itemize}

    \psection{Регулярная грамматика исходного языка}
    \noindent
    $\varepsilon$ -- пустая строка\\ \\
    {\bf program}: predicates goals\\
    {\bf predicates}: $\varepsilon$ | predicates predicate | predicate\\
    {\bf prediate}: identifier open\_par close\_par dot | identifier open\_par close\_par define statements dot | 
        identifier open\_par arguments close\_par dot | identifier open\_par arguments close\_par define statements dot\\
    {\bf arguments}: argument | arguments comma argument\\
    {\bf argument}: variable | number | string | list\\
    {\bf list}: open\_brc close\_brc | open\_brc sequence close\_brc\\
    {\bf sequence}: term | sequence comma term\\
    {\bf term}: variable | number | string | list\\
    {\bf statements}: statement | statements comma statement\\
    {\bf statement}: cut | line | write open\_par sequence close\_par | succ open\_par variable comma variable close\_par |
        identifier open\_par arguments close\_par\\
    {\bf goals}: $\varepsilon$ | goal | goals goal\\
    {\bf goal}: init open\_par statement close\_par dot
    \vskip 6em
    \section{Детерминированная автоматная модель синтаксического анализатора}
    Для выполнения поставленной задачи был выбран LR(1)-анализатор.
	Для данного анализатора строится конечный детерминированный автомат с магазинной памятью.
	L обозначает то, что он читает входящий поток слева направо.
	R указывает на то, что разбор будет происходить в обратном порядке, т.е. снизу вверх.
	Индекс показывает то, что анализатор при принятии решений будет видеть следующую лексему.
	Было принято решение реализовывать автомат посредством утилит Flex и Bison.
    
    \psection{Грамматика свойств}
    \noindent
    Свойства:\\
    0 --- нейтральное\\
    1 --- объявление правила\\
    2 --- использование правила\\
    3 --- использование необъявленного предиката \\ \\
    {\bf program}: predicates goals\\
	\begin{tabular}{c c | c}
		\multicolumn{2}{c |}{S} & $\mu$ \\ 
		\hline
		 0 & 0 & 0 \\
		 0 & 2 & 2 \\
		 1 & 0 & 0 \\
		 1 & 2 & 2 \\
		 2 & 0 & 0 \\
		 2 & 2 & 2 \\
		 3 & 0 & 3 \\
		 0 & 3 & 3 \\
		 3 & 3 & 3 \\
	\end{tabular} 

    \noindent
    {\bf predicates}: $\varepsilon$\\
 	\begin{tabular}{c | c}
		\multicolumn{1}{c |}{S} & $\mu$ \\ 
		\hline
		 0 & 0 \\ 
	\end{tabular} 

    \noindent
    {\bf predicates}: predicate\\
 	\begin{tabular}{c | c}
		\multicolumn{1}{c |}{S} & $\mu$ \\ 
		\hline
		 1 & 1 \\
	\end{tabular} 

    \noindent
    {\bf predicates}: predicates predicate\\
	\begin{tabular}{c c | c}
		\multicolumn{2}{c |}{S} & $\mu$ \\ 
		\hline
		 1 & 1 & 1 \\
		 1 & 2 & 2 \\
	\end{tabular} 

    \noindent
    {\bf prediate}: identifier open\_par close\_par dot\\
	\begin{tabular}{c c c c | c}
		\multicolumn{4}{c |}{S} & $\mu$ \\ 
		\hline
        0 & 0 & 0 & 0 & 1 \\
	\end{tabular} 

    \noindent
    {\bf prediate}: identifier open\_par close\_par define statements dot\\
	\begin{tabular}{c c c c c c | c}
		\multicolumn{6}{c |}{S} & $\mu$ \\ 
		\hline
        0 & 0 & 0 & 0 & 2 & 0 & 2 \\
        0 & 0 & 0 & 0 & 3 & 0 & 3 \\
	\end{tabular} 

    \noindent
    {\bf prediate}: identifier open\_par arguments close\_par dot\\
 	\begin{tabular}{c c c c c | c}
		\multicolumn{5}{c |}{S} & $\mu$ \\ 
		\hline
        0 & 0 & 0 & 0 & 0 & 1 \\
	\end{tabular} 

    \noindent
    {\bf prediate}: identifier open\_par arguments close\_par define statements dot\\
 	\begin{tabular}{c c c c c c c | c}
		\multicolumn{7}{c |}{S} & $\mu$ \\ 
		\hline
        0 & 0 & 0 & 0 & 0 & 2 & 0 & 2 \\
        0 & 0 & 0 & 0 & 0 & 3 & 0 & 3 \\
	\end{tabular} 

    \noindent
    {\bf statements}: statement\\
	\begin{tabular}{c | c}
		\multicolumn{1}{c |}{S} & $\mu$ \\ 
		\hline
		 2 & 2 \\
		 3 & 3 \\
	\end{tabular} 

    \noindent
    {\bf statements}: statements comma statement\\
	\begin{tabular}{c c c | c}
		\multicolumn{3}{c |}{S} & $\mu$ \\ 
		\hline
        2 & 0 & 2 & 2 \\
        2 & 0 & 3 & 3 \\
        3 & 0 & 2 & 3 \\
        3 & 0 & 3 & 3 \\
	\end{tabular} 

    \noindent
    {\bf statement}: cut | line | write open\_par sequence close\_par | succ open\_par variable comma variable close\_par\\
    % mu = 2
    \begin{tabular}{c | c}
        S & $\mu$ \\
        \hline
        $\cdots$ & 2 \\
    \end{tabular} 

    \noindent
    {\bf statement}: identifier open\_par arguments close\_par\\
    \begin{tabular}{c c c c | c}
        \multicolumn{4}{c |}{S} & $\mu$ \\
        \hline
        0 & 0 & 0 & 0 & 2 \\
        0 & 0 & 3 & 0 & 3 \\
	\end{tabular} 

    \noindent
    {\bf goals}: goal\\
	\begin{tabular}{c | c}
		\multicolumn{1}{c |}{S} & $\mu$ \\ 
		\hline
        0 & 0 \\
		2 & 2 \\
		3 & 3 \\
	\end{tabular} 
    
    \noindent
    {\bf goals}: goals goal\\
	\begin{tabular}{c c | c}
		\multicolumn{2}{c |}{S} & $\mu$ \\ 
		\hline
		 0 & 0 & 0 \\
		 0 & 2 & 2 \\
		 0 & 3 & 3 \\
		 2 & 0 & 2 \\
		 2 & 2 & 2 \\
		 2 & 3 & 3 \\
		 3 & 0 & 3 \\
		 3 & 2 & 3 \\
		 3 & 3 & 3 \\
	\end{tabular} 

    \noindent
    {\bf goal}: init open\_par statement close\_par dot\\
    \begin{tabular}{c c c c c | c}
		\multicolumn{5}{c |}{S} & $\mu$ \\ 
		\hline
        0 & 0 & 2 & 0 & 0 & 2 \\
        0 & 0 & 3 & 0 & 0 & 3 \\
	\end{tabular} \\

    \noindent
    {\bf arguments}: argument\\
    {\bf arguments}: arguments comma argument\\
    {\bf argument}: variable | number | string | list\\
    {\bf list}: open\_brc close\_brc | open\_brc sequence close\_brc\\
    {\bf sequence}: term | sequence comma term\\
    {\bf term}: variable | number | string | list\\
    {\bf goals}: $\varepsilon$ \\
     \begin{tabular}{c | c}
        S & $\mu$ \\
        \hline
        $\cdots$ & 0 \\
    \end{tabular}    
 
    \psection{Структура разработанной программы}
    \noindent
    Разработанная программа состоит из следующих файлов:
    \begin{itemize}[leftmargin=*] \itemsep1pt
        \item[---] \verb|scanner.l| -- определения токенов и лексем для генератора GNU flex;
        \item[---] \verb|parser.y| -- описание грамматики свойств для программы GNU bison;
        \item[---] \verb|struct.h| -- заголовочный файл языка С, содержащий объявления вспомогательных функций и структур;
        \item[---] \verb|struct.c| -- файл языка С, содержащий определения вспомогательных функций и точку входа программы. 
    \end{itemize}
    Описание вспомогательных функций и структур:
    \begin{itemize}[leftmargin=*] \itemsep1pt
        \item[---] \verb|struct List| -- структура (список) для хранения идентификаторов и их свойств;
        \item[---] \verb|void yyerror(char * start, ...)| -- вывод сообщений об ошибках;
        \item[---] \verb|struct List * add_identifier(struct List * list, char * name, int args)| -- добавление идентификатора в список;
        \item[---] \verb|struct List * check_identifier(struct List * list, char * name)|\\-- проверка наличия идентификатора в списке;
        \item[---] \verb|struct List * delete_table(struct List * list)|\\-- освобождение выделенной под структуру памяти; 
        \item[---] \verb|char *identifier_parse(char *str)|\\-- извлечение имени идентификатора из строки программы.
    \end{itemize}

    \psection{Результаты тестирования} {
    \lstset{numberstyle=\footnotesize\ttfamily, numbers=left, 
    frame=single, language=Prolog, breaklines=true, showstringspaces=false,basicstyle=\footnotesize\ttfamily}
   
    {\centering \bf Тест №1\\}
    \noindent
    Входной файл:
    \lstinputlisting{../src/0.pl}
    Вывод программы:\\
    \verb|OK.|\\

    {\centering \bf Тест №2\\}
    \noindent
    Входной файл:
    \lstinputlisting{../src/1.pl}
    Вывод программы:\\
    \verb|error: using undefined predicate, line 6|\\
    \verb|error: using undefined predicate, line 8|\\
    \verb|Not OK.|\\

    {\centering \bf Тест №3\\}
    \noindent
    Входной файл:
    \lstinputlisting{../src/2.pl}
    Вывод программы:\\
    \verb|error: wrong arguments number, line 7|\\
    \verb|error: using undefined predicate, line 10|\\
    \verb|Not OK.|\\

    {\centering \bf Тест №4\\}
    \noindent
    Входной файл:
    \lstinputlisting{../src/3.pl}
    Вывод программы:\\
    \verb|error: syntax error, line 6|\\
    \verb|Not OK.|\\

    {\centering \bf Тест №5\\}
    \noindent
    Входной файл:
    \lstinputlisting{../src/4.pl}
    Вывод программы:\\
    \verb|error: syntax error, line 10|\\
    \verb|Not OK.|\\

    }

    \psection{Руководство пользователя}
    Для составления корректных входных данных пользователю следует ознакомиться с разделом
    №~\ref{sec:task} <<Постановка задачи>>, в которой в общих чертах описано допустимое подмножество
    языка Prolog.\\
    Запуск программы осуществляется посредством исполняемого файла <<./apl>>. Программа работает со
    стандартным потоком ввода. Таким образом, существует два варианта использования программы:
    построчный ввод входных данных в консоли или использование утилиты cat и перенаправления входного
    файла в стандартный поток ввода.\\
    Пример: \texttt{cat mytest.pl | ./apl}
    В ходе работы программы пользователь может получать сообщения об ошибках с указанием номера строки,
    а так же конечный вердикт: <<OK>>, если исходные данные корректны, <<Not OK>> -- в ином случае.\\
    
    \vskip 6em

    \section*{Заключение}
    \addcontentsline{toc}{section}{Заключение}
    В ходе выполнения курсового проекта я закрепил полученные в течение курса ТЯПиМТ знания,
    подробнее изучил процесс компиляции и интерпретации языка Prolog. Написанная программа выполняет
    поставленную задачу корректно.\\ \\
    К её достоинствам можно отнести:
	\begin{itemize} [leftmargin=*] \itemsep1pt
        \item[---] Грамматика составлена с учетом нюансов языка Prolog;
        \item[---] Программа безошибочно отлавливает как синтаксические, так и семантические ошибки;
        \item[---] Программа написана на С, что гарантирует ей корректное построение автоматов в пакетах Flex и Bison.
	\end{itemize}
	К недостаткам можно отнести:
	\begin{itemize} [leftmargin=*] \itemsep1pt
        \item[---] Допустимое подмножество покрывает лишь некоторые базовые конструкции языка Prolog, в силу того, что
            большинство базовых операций реализованно встроенными предикатами и для них необходимы дополнительные
            проверки;
        \item[---] Нет возможности проверять переменные, использующиеся как аргументы, так как такая проверка может быть
            произведена лишь во время интерпретации.
	\end{itemize}

    \unsection{Приложение}
    \lstset{numberstyle=\footnotesize\ttfamily, numbers=left, 
    frame=single, language=C, breaklines=true, showstringspaces=false,basicstyle=\footnotesize\ttfamily}
    \noindent \texttt{Makefile:}    
    \lstinputlisting{../Makefile}
    \texttt{scanner.l:}
    \lstinputlisting{../.src/scanner.l}
    \newpage
    \texttt{parser.y:}
    \lstinputlisting{../.src/parser.y}
    \newpage
    \texttt{struct.h:}
    \lstinputlisting{../.src/struct.h}
    \texttt{struct.c:}
    \lstinputlisting{../.src/struct.c}

%% --------------------------------------------------------    
\end{document}
